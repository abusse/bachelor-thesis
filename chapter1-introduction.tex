%!TEX root = bachelor.tex

\chapter{Einleitung}%
\label{chap:intro}

Spätestens seit der Finanzkrise ab dem Jahr 2007 sind die Möglichkeit und Folgen einer systemischen Krise in das Zentrum der Aufmerksamkeit von Öffentlichkeit und Forschung getreten. Eine systemische Krise ist eine grundlegenden Gefahr für jede Volkswirtschaft und hat dramatische Folgen, die in einer signifikanten Anzahl von Firmeninsolvenzen und massiver Verringerung des Wohlstandes in der betroffenen Volkswirtschaft führen kann. Ein Beispiel für die Konsequenzen einer solchen Krise konnten und können bis heute in Griechenland beobachtet werden (\vgl{}~\cite{Papadimitriou-2013}).

Eine zentrale Ursache für eine systemische Krise kann die gleichzeitige Insolvenz mehrere Banken sein, die zum Kollaps des gesamten Bankensektors führt, was wiederum die gesamte Geschäftstätigkeit innerhalb einer Volkswirtschaft massiv einschränkt. Somit kann dieses Szenario unstrittig als ein systemisches Risiko eingestuft werden. Als ein Mittel zur Verringerung des Risikos wird im Allgemeinen Diversifikation gesehen. Jedoch bezieht sich die Verringerung des Risikos meist nur auf das individuelle Risiko einer Bank und nicht auf das Risiko bezüglich einer systemischen Krise. Dieser Aspekt wird \ua{} von \textcite{Wagner-2010} genauer untersucht. In seiner Arbeit präsentiert \citeauthor{Wagner-2010} das Model eines Bankensektors bestehend aus zwei Banken. Er zeigt, dass durch die Diversifikation zwar das individuelle Risiko einer Bank insolvent zu werden, durch die Diversifikation sinkt, gleichzeitig aber das Risiko einer systemischen Krise steigt. Er zeigt ferner, dass ohne Regulierung die Banken in seinem Modell zu einer Überdiversifizierung neigen.

\Citeauthor{Wagner-2010} steht dabei mit seiner Feststellung, dass Diversifikation und insbesondere vollständige Diversifikation nicht zwangsläufig wünschenswert ist, nicht alleine. \Textcite{Goldstein-2004} \zb{} untersuchen die Effekte von Ansteckungen und selbsterfüllende Prophezeiungen bezüglich Krisen zwischen verschiedenen Märkten. Auch sie kommen zu dem Schluss, dass eine vollständige Diversifizierung nicht immer von Vorteil ist und das Risiko einer umfangreichen Krise erhöhen kann. Zu einem ähnlichen Schluss kommen \textcite{Danielsson-2003}. Sie untersuchen in ihrer Arbeit den Einfluss von Handelsregulierungen bezüglich Finanzmarktderivaten und wenden Ihre Erkenntnisse auf die Krisen am Schwarzen Montag 1987 und die Krise im russischen Finanzmarkt 1998 an. Neben der Erkenntnis, dass eine zwanghafte Diversifikation volkswirtschaftlich nicht optimal ist, stellen sie weiter fest, dass allgemeine Regulierungen, die für alle Marktteilnehmer und insbesondere Banken gleich sind, ebenfalls als suboptimal angesehen werden können. \Textcite{Stiglitz-2010a,Stiglitz-2010b} stellt ebenfalls fest, dass eine vollständige Diversifikation nicht wünschenswert ist. Er führt Problematiken jedoch primär auf Ansteckungen zwischen den Banken \bzw{} Märkten zurück. \Textcite{Ibragimov-2011} führen schlussendlich die Arbeit von \citeauthor{Wagner-2010} fort und vervollständigen diese insbesondere in Bezug auf alternative Risikowahrscheinlichkeiten \bzw{} Wahrscheinlichkeitsverteilungen. 

Diversifizierung kann jedoch auch aus anderen Aspekten zu systemischen Krisen führen. \Textcite{Battiston-2012} und \textcite{Teteryatnikova-2014} \zb{} untersuchen Bankennetzwerke, wobei letztere von beiden Arbeiten davon ausgeht, dass einige wenige Banken führend sind und mehr Verbindungen haben als andere. Auch diese Arbeiten kommen zu dem Schluss, dass Diversifizierung zwar das Risiko systemischer Krisen verhindern kann und den Gewinn von Banken erhöhen kann, jedoch stellen auch sie fest, dass eine vollständige Diversifikation nicht zwingend optimal ist.

Schlussendlich kann die Diversifizierung auch negative Folgen abseits des steigenden Risikos einer systemischen Krise haben. \Textcite{Laeven-2007} diskutieren \zb{} in Ihrer Arbeit, dass Diversifikation den Wert einer Bank verringern kann. Sie führen dies in erster Linie auf mögliche Interessenskonflikte bezüglich des Prinzipal-Agenten-Problems zurück, welche bei einer großen Anzahl von Aktivitäten entstehen können. \Textcite{Elsas-2010} widersprechen jedoch diesen Ergebnissen und kommen basierend auf Daten aus neun Ländern in der Zeit von 1996 bis 2008 zu dem Schluss, dass Diversifizierung den Wert von Banken erhöht. Sie führen den Widerspruch zur Arbeit von \citeauthor{Laeven-2007} auf verschiedene Methoden zur Ermittlung des Wertes einer Bank zurück.

Auf Grund der hohen Relevanz wird in dieser Arbeit das von \citeauthor{Wagner-2010} vorgestellte Model dargestellt, ausführlich diskutiert und kritisch gewürdigt. Hierfür wird zunächst in \cref{chap:basic} das Basismodel ohne Diversifikation vorgestellt, auf dem \citeauthor{Wagner-2010} aufbaut. \Cref{sec:div:model} fährt fort, \citeauthor{Wagner-2010}s Erweiterung des Basismodells um Diversifikation zu erläutern. Es wird dabei auch der Aspekt der Ansteckung berücksichtigt. In \cref{chap:advanced} wird \citeauthor{Wagner-2010}s Modell dahingehend kritisch betrachtet, als er bestimmte Randbedingung bei seiner Analyse vernachlässigt hat. Im Anschluss wird in \cref{chap:other} die Analyse von Banksektoren, welche keine Gleichverteilung der Ergebnisse der Investments aufweisen, präsentiert. \Cref{chap:conclusion} schließt schlussendlich diese Arbeit mit einer Zusammenfassung und einem Ausblick für zukünftige Anknüpfungspunkte.
