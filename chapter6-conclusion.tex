%!TEX root = bachelor.tex

\chapter{Zusammenfassung und Ausblick}%
\label{chap:conclusion}

Die Diskussion von \citeauthor{Wagner-2010}s Arbeit hat gezeigt, dass durch die Diversifizierung von Banken das Risikio indivueller Krisen \bzw{} eines Krisenfalls im Allgemeinen gesenkt werden kann. Dies geht jedoch mit den Gefahr einher, dass das Risiko für systemische Krisen steigt. Dies lässt sich dadurch erklären, dass die Banken durch die Diversifizierung sich strukturell angleichen, wodurch sie als Konsequenz jeweils auch den selben Risiken ausgesetzt sind. Für einen vollständig diversifizierten Markt würde somit eine Krise sofort eine systemische Krise bedeuten.

Die Ausarbeitung von \citeauthor{Wagner-2010} wurde im Rahmen dieser Arbeit kritisch hinterfragt und ergänzt. Zunächst wurde in \cref{chap:advanced} das von \citeauthor{Wagner-2010} aufgestellte Modell genauer untersucht und es wurden zusätzliche Fälle betrachtet, die bei \citeauthor{Wagner-2010} auf Grund von Vereinfachungen keine Berücksichtigung fanden. Da insbesondere von \textcite{Ibragimov-2011} kritisiert wurde, dass \citeauthor{Wagner-2010} lediglich eine Gleichverteilung betrachtet, erfolgte in \cref{chap:other} nach dieser Präzisierung eine Betrachtung von Märkten, bei denen die Ergebnisse der Investments nicht einer Gleichverteilung folgen. Prinzipiell ist \citeauthor{Wagner-2010}s Modell auf jeden Markt anwendbar, in dem die Investmentrückläufe einer kompakten Verteilung folgen. \Cref{sec:other:piece_wise} kam jedoch am Beispiel einer einfachen Dreiecksverteilung zu der Erkentniss, dass eine Analys eines Marktes aufwendig ist, in dem die Verteilung der Investments stückweise definiert ist. In \cref{sec:other:numeric} wurde \citeauthor{Wagner-2010}s Modell mit komplexeren Verteilunsfunktionen numerisch analysiert. Auf Grund der Erkentniss aus \cref{sec:other:piece_wise} wurde dabei auf Verteilungen zurückggegriffen, die im fraglichen Invervall kontinuierlich definiert sind. Die Analyse hat gezeigt, dass die Ergebnisse von \citeauthor{Wagner-2010} sich auch grundsäztlich bei anderen Verteilungen wiederfinden. Jedoch unterscheidet sich der Grad der Überdiversifizierung teilweise signifikant. Ferner wurde deutlich, dass komplexere Verteilungen einen signifikanten Aufwand bezüglich der Berechnung erfordern, was die praktische Anwendung \citeauthor{Wagner-2010}s Models weiter erschwert.

Trotz der intensiven Auseinandersetzung in dieser Arbeit, existieren weitere Anknüfungspunkte zur Erweiterung der Arbeit von \citeauthor{Wagner-2010}. Zunächst wäre ein Betrachtung sinnvoll, die halboffene Verteilungen für die Ergebnisse der Investments annimmt. Dies würde auch die Fallunterscheidung, wie sie in \cref{chap:advanced} vorgenommen wurde, unnötig machen. Ferner kann der Verlust bei Übertragung des Investments auf eine andere Bank oder bei der Liquidierung ausfühlicher diskutiert werden. So könnte man annhemen, dass der Verlust bei Übertragung eines Investments an eine andere Bank geringer ausfällt, wenn die andere Bank durch Diversifizierung auch in dem Bereich tätig ist. Somit wäre die Höhe des Verlusts vom Diversifizierungsgrad der anderen Bank abhängig. Insbesondere in diesem Zusammenhang lässt sich außerdem untersuchen, in wie fern es sinnvoll ist, dass die Kosten für die Liquidierung linear von der Höhe der Kosten des Transfers des Investments abhängen. Es erscheit \zb{} sinnvoll, dass die Liquidierungskosten einen fixen Anteil haben, wenn bei Diversifizierung keine oder sehr geringe Kosten beim Transfer entstehen.