%!TEX root = bachelor.tex

\makeglossaries
\renewcommand*{\glossaryname}{Symbolverzeichnis}
\deftranslation{Glossary}{Symbolverzeichnis}
\renewcommand*{\glspostdescription}{}

\setlength{\glslistdottedwidth}{.14\linewidth}

\newglossaryentry{symb:pis}{
name=$\pi^S$,
description={Wahrscheinlichkeit einer systemischen Krise.},
sort=symbolt
}

\newglossaryentry{symb:pi1}{
name=$\pi^I_i$,
description={Wahrscheinlichkeit einer individuellen Krise der Bank $i$.},
sort=symbolt
}

\newglossaryentry{symb:pii}{
name=$\pi_i$,
description={Wahrscheinlichkeit einer Krise der Bank $i$.},
sort=symbolt
}

\newglossaryentry{symb:d}{
name=$d$,
description={Anteil der Einlagen am Gesamtkapital einer Bank.},
sort=symbolt
}

\newglossaryentry{symb:x}{
name=$x$,
description={Tatsächliche Realisierung des Werts der Anlage $X$.},
sort=symbolt
}

\newglossaryentry{symb:y}{
name=$y$,
description={Tatsächliche Realisierung des Werts der Anlage $Y$.},
sort=symbolt
}

\newglossaryentry{symb:c}{
name=$c$,
description={Verlust beim Transfer einer Anlage zwischen Banken.},
sort=symbolt
}

\newglossaryentry{symb:q}{
name=$q$,
description={Verlustfaktor bei vorzeitiger Liquidierung einer Anlage.},
sort=symbolt
}

\newglossaryentry{symb:vi}{
name=$v_i$,
description={Wert der Bank $i$.},
sort=symbolt
}

\newglossaryentry{symb:phi}{
name=$\phi$,
description={Eine Wahrscheinlichkeitsdichtefunktion für eine beliebige Verteilung mit kompakten Intervall.},
sort=symbolt
}

\newglossaryentry{symb:s}{
name=$s$,
description={Höchstmögliche Realisierung des Werts einer Anlage.},
sort=symbolt
}

\newglossaryentry{symb:ri}{
name=$r_i$,
description={Diversifizierungsgrad der Bank $i$.},
sort=symbolt
}

\newglossaryentry{symb:roptall}{
name=$r^\ast$,
description={Volkswirtschaftlich optimaler Diversifizierungsgrad der Banken.},
sort=symbolt
}

\newglossaryentry{symb:roptone}{
name=$r^E$,
description={Individuell optimaler Diversifizierungsgrad einer Bank.},
sort=symbolt
}
