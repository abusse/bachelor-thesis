%!TEX root = bachelor.tex

\begingroup
\singlespacing{}
\chapter*{Zusammenfassung}
\endgroup

Systemische Krisen stellen eine substantielle Gefahr für jede Volkswirtschaft dar. Aus diesem Grund ist die Untersuchung der Risiken, die zu solch einer Krise führen, unabdingbar. Auslöser einer systemischen Krise kann die gleichzeitige Insolvenz aller oder mehrerer Banken sein. Obwohl Diversifizierung das individuelle Risiko einer Bank bezüglich einer Insolvenz verringern kann, kann diese Diversifizierung das Risiko einer systemischen Krise vergrößern. Diese Problematik wird in der Veröffentlichung von \emph{Wolf Wagner} mit dem Titel \mkbibquote{Diversification at Financial Institutions and Systemic Crises} detailliert diskutiert.

Die vorliegende Arbeit stellt die Erkenntnisse von Wagner ausführlich dar. Es wird dabei anhand eines 2-Banken-Modells erläutert, wie und in welchem Ausmaß Diversifikation zwar das individuelle Risiko einer Bank verringern kann, gleichzeitig jedoch das Risiko einer systemischen Krise steigert. Wagner kommt zu dem Schluss, dass eine vollständige Diversifikation aus diesem Grund nicht optimal sein kann und quantifiziert den optimalen Diversifizierungsgrad für konkrete Beispiele. Er kommt dabei zu der Erkenntnis, dass ohne eine Regulierung im Bankensektor ein zu hoher Diversifizierungsgrad präferiert wird.

Wagners Ausarbeitung wird im Rahmen dieser Arbeit in drei Aspekten kritisch hinterfragt. Zunächst werden Modellparameter untersucht, die bei Wagner keine Rücksicht fanden. Dadurch wird seine Modellanalyse komplettiert. Als zweiter Aspekt der kritischen Untersuchung wird gezeigt, dass Wagners Modell auch auf komplexere Wahrscheinlichkeitsverteilungen angewandt werden kann. Dabei werden insbesondere etwaige Probleme diskutiert, die bei der Nutzung von abschnittsweise definierten Verteilungsfunktionen entstehen. Schlussendlich findet im Rahmen dieser Arbeit eine numerische Analyse eines Bankenmarktes statt. Durch die Analyse werden Wagners Aussagen und Ergebnisse weiter untermauert.

\cleardoublepage
\begingroup
\singlespacing{}
\chapter*{Abstract}
\endgroup

\begin{otherlanguage*}{english}

Systemic crises are a great threat to every economy. Hence, it is inevitable to analyse the risks that lead to such a crisis. If several or all banks go bankrupt simultaneously, a systemic crisis might emerge. Even though diversification may minimize the risk of going bankrupt for single banks, it can also increase the chance of a systemic crisis. \emph{Wolf Wagner} discusses this problem in detail in his paper \mkbibquote{Diversification at Financial Institutions and Systemic Crises}. 

This thesis aims at presenting Wagner’s findings elaborately. Using a two financial institutions model, I will explain how and to what extent diversification is able to decrease the individual risk of a bank, yet at the same time increases the risk of a systemic crisis. Wagner concludes that a total diversification is not the best solution for this reason and furthermore quantifies the optimum grade of diversification for concrete examples.

Wagner’s work will be analysed critically with regard to three aspects. First, I will analyse the parameters that Wagner has not considered, hence completing the analysis of his model. Second, the critical examination will show that Wagner’s model is also applicable to probability distributions that are more complex. The discussion will focus especially on problems that emerge from the use of sectionally defined functions describing probability distributions. Third and last, the financial market will be examined numerically. This examination will confirm Wagner’s conclusions and results. 

\end{otherlanguage*}
